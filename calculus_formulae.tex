\documentclass{article}
%\usepackage{amsmath}
\usepackage[fleqn]{amsmath}
\usepackage{geometry}
 %\usepackage{showframe} %This line can be used to clearly show the new margins
\newgeometry{vmargin={15mm}, hmargin={12mm,15mm}}
\title{Calculus Formulae}

\begin{document}
\maketitle

\section{Some Integrals}

\subsection{Integrating Hyperbolic Functions}
\begin{minipage}{.5\linewidth}
\begin{align*}
    \int sinh(ax) dx &= \frac{1}{a} cosh \ ax + C \\
\end{align*}
\begin{align*}
    \int sinh^2(ax) dx &= \frac{1}{4a} sinh \ 2ax - \frac{x}{2} + C
\end{align*}
\begin{align*}
    \int x \ sinh(ax) dx &= \frac{1}{a} x \ cosh \ ax - \frac{1}{a^2} sinh \ ax + C
\end{align*}
\end{minipage}%%%%%%%%%%%%%%%%%%%%%%  
\begin{minipage}{.5\linewidth}
\begin{align*}
    \int cosh(ax) dx &= \frac{1}{a}sinh \ ax + C
\end{align*}
\begin{align*}
    \int cosh^2(ax) dx &= \frac{1}{4a} sinh \ 2ax - \frac{x}{2} + C
\end{align*}
\begin{align*}
    \int x \ cosh(ax) dx &= \frac{1}{a} x \ sinh \ ax - \frac{1}{a^2} cosh \ ax + C
\end{align*}
\end{minipage}  


\begin{align*}
    \int sinh^n(ax) dx &= \frac{1}{na} (sinh^{n-1} \ 2ax)(cosh ax) - \frac{n-1}{n} \int sinh^{n-2}(ax) dx \\
    \int cosh^n(ax) dx &= \frac{1}{na} (cosh^{n-1} \ 2ax)(sinh ax) + \frac{n-1}{n} \int cosh^{n-2}(ax) dx \\
    \int tanh(ax) dx &= \frac{1}{a} ln(cosh \ ax) + C \\
    \int coth(ax) dx &= \frac{1}{a} ln(sinh \ ax) + C \\
\end{align*}
\subsection{Integrating to Hyperbolic Functions}

\begin{align*}
    \int \frac{1}{\sqrt{a^2+x^2}} dx = arcsinh(\frac{x}{a}) + C
\end{align*}

\begin{align*}
    \int \frac{1}{\sqrt{x^2-a^2}} dx = arccosh(\frac{x}{a}) + C
\end{align*}

\begin{align*}
    \int \frac{1}{a^2-x^2} dx = \frac{1}{a} \ arctanh(\frac{x}{a}) + C \qquad (x^2<a^2)
\end{align*}

\begin{align*}
    \int \frac{1}{a^2-x^2} dx = \frac{1}{a} \ arccoth(\frac{x}{a}) + C \qquad (x^2>a^2)
\end{align*}

\begin{align*}
    \int \frac{1}{x\sqrt{a^2-x^2}} dx = -\frac{1}{a} \ arcsech(\frac{x}{a}) + C
\end{align*}

\begin{align*}
    \int \frac{1}{x\sqrt{a^2+x^2}} dx = -\frac{1}{a} \ arcsech|\frac{x}{a}| + C
\end{align*}
\newline

\textbf{Integrating Trigs}

\begin{align*}
    \int sin^2(ax) dx = \frac{x}{2} - \frac{1}{4a} sin \ 2ax + C
\end{align*}

\begin{align*}
    \int x \ sin(ax) dx = \frac{sin \ ax}{a^2} - \frac{x \ cos \ ax}{a}+ C
\end{align*}

\begin{align*}
    \int cos^2(ax) dx = \frac{x}{2} + \frac{1}{4a} sin \ 2ax + C
\end{align*}

\begin{align*}
    \int x \ cos(ax) dx = \frac{cos \ ax}{a^2} - \frac{x \ sin \ ax}{a}+ C
\end{align*}
\textbf{Integrating to Trigs}  
% \begin{minipage}
    
% \end{minipage}
\newpage


\section{Some Vector Calculus}
\Large \textbf{Tensors}
\begin{align*}
    \varepsilon_{ijk} \varepsilon_{klm} = \delta_{il} \delta_{jm} - \delta_{im} \delta_{jl}
\end{align*}
\begin{align*}
    \varepsilon_{ijk} \varepsilon_{ilm} = \delta_{jl} \delta_{km} - \delta_{jm} \delta_{kl}
\end{align*}
\newline


\noindent
\Large{\textbf{Grad, Div, Curl}}
\newline

\textbf{Directional Derivative}
\begin{align*}
    \frac{\partial \phi}{\partial s} &= \frac{\partial \phi}{\partial n} (\hat{n} \cdot  \hat{s}) \\
                                     &= \mathbf{\hat{s}} \cdot \nabla \phi
\end{align*}

\textbf{Laplacian}
\begin{align*}
    \nabla^2 \phi &= div(\nabla \phi) \\
                  &= \frac{\partial^2 \phi }{\partial x_i^2}
\end{align*}  
\noindent
\textbf{Green's Theorem}  
\begin{flushleft}
Closed curve $C$, $L$ and $M$ are continuously differentiable over $R$. 
\end{flushleft} 
\begin{align*}
    \oint_{C} \  (L \ dx + M \ dy) &= \int_{R} (\frac{\partial M}{\partial x}-\frac{\partial L}{\partial y}) \ dxdy \\
    \oint_{C} \  \mathbf{F} \cdot \mathbf{\hat{n}} \ ds &= \int_R div \mathbf{F} \  dxdy \\
\end{align*}
\newline
\newline

\noindent
\textbf{Divergence Theorem}  

Volume $\tau$, closed surface $S$, outward normal $\mathbf{\hat{n}}$ 
continuous derivatives.
\begin{align*}
    \int_S \mathbf{A} \cdot \mathbf{\hat{n}} \ dS= \int_{\tau} \text{div} \mathbf{A} \ d\tau
\end{align*}  
\newline

\noindent
\textbf{Gauss' Flux Theorem}  

\begin{flushleft}
    $S$ closed surface with outward unit normal $\mathbf{\hat{n}}$, $O$ is the origin of the coordinate system.
\end{flushleft} 
\begin{equation}
    \int_S \frac{\mathbf{r} \cdot \mathbf{\hat{n}}}{r^3} \ dS= 
    \begin{cases}
        & 0, \ \text{if $O$ is outside} \\
        & 4\pi, \ otherwise \\
    \end{cases}
\end{equation}  
\newline

\noindent
\textbf{Stokes Theorem}
\begin{flushleft}
    $S$ an open surface, simple closed curve $\gamma$, $\mathbf{A}$ with continuous partial derivatives, 
    outward normal $\mathbf{\hat{n}}$ determined by right-hand rule
\end{flushleft} 
\begin{align*}
    \oint_{\gamma} \  \mathbf{A} \cdot \mathbf{dr} = \int_{S} curl \mathbf{A} \cdot \mathbf{\hat{n}} \ dS
\end{align*}  
\newline

\noindent
\Large \textbf{Curvilinear coordinates}  
\newline
\newline

\textbf{Cylindrical Coordinates}  

The Jacobian determinant is $r$.  

$x = r cos \phi \qquad y = rsin \phi \qquad z=z$
\begin{align*}
    h_1 &= 1 \\
    h_2 &= r \\ 
    h_3 &= 1 \\
\end{align*}

\indent Gradient, Divergence, Curl, and Laplacian
\begin{align*}
    \nabla &= \hat{r} \frac{\partial}{\partial r} + \frac{\hat{\phi}}{r} \frac{\partial}{\partial \phi} + \hat{k} \frac{\partial}{\partial z} \\
    \nabla \cdot \mathbf{A} &= \frac{\partial A_1}{\partial r} + \frac{A_1}{r} + \frac{1}{r} \frac{\partial A_2}{\partial \phi} + \frac{\partial A_3}{\partial z} \\
    curl \mathbf{A} &=\frac{1}{r} \begin{vmatrix}
        \mathbf{\hat{r}} & r \hat{\phi} & \mathbf{\hat{k}} \\ 
        \frac{\partial}{\partial r} & \frac{\partial}{\partial \phi} & \frac{\partial}{\partial z} \\ 
        A_1 & rA_2 & A_3 \\ 
        \end{vmatrix} \\
    \nabla^2 \Phi &= \frac{\partial^2 \Phi}{\partial r^2} + \frac{1}{r} \frac{\partial \Phi}{\partial r} + \frac{1}{r^2} \frac{\partial^2 \Phi}{\partial \phi} + \frac{1}{r} \frac{\partial^2 \Phi}{\partial z}
\end{align*}

\textbf{Spherical Polar}  

The Jacobian determinant is $r^2 sin \theta$.  

$x = r sin \theta cos \phi \qquad y = rsin \theta sin \phi \qquad z=r cos\theta$  
\begin{align*}
h_1 &= 1 \\
h_2 &= r \\ 
h_3 &= rsin\theta \\
\end{align*}

\textbf{Grad, Div, Curl and Laplacian in Curvilinear}  

\begin{align*}
    \nabla &= \frac{\hat{e_1}}{h_1}  \frac{\partial}{\partial u_1} + \frac{\hat{e_2}}{h_2} \frac{\partial}{\partial u_2} + \frac{\hat{e_3}}{h_3}  \frac{\partial}{\partial u_3} \\
\end{align*}



\section{Differential Equations}

\textbf{Euler-Cauchy}
\begin{align*}
    \mathcal{L}[y] = \beta_{k}x^k \frac{d^ky}{dx^k} + \dots + \beta_{1} x \frac{dy}{dx} = f(x) 
\end{align*}
Try substitution $x=e^z$:
\begin{align*}
    \frac{dy}{dx} &= \frac{1}{x} \frac{dy}{dz} \\
    \frac{d^2y}{dx^2} &= \frac{1}{x^2} [\frac{d^2y}{dz^2}-\frac{dy}{dz}] \\
    \frac{d^3y}{dx^3} &= \frac{1}{x^3} [\frac{d^3y}{dz^3}-3 \ \frac{d^2y}{dz^2}+2 \ \frac{dy}{dz}]
\end{align*}
Note the $a(a-1)(a-2)$ factorial like pattern here.

\end{document}