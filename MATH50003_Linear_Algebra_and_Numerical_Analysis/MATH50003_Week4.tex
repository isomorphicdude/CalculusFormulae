% Options for packages loaded elsewhere
\PassOptionsToPackage{unicode}{hyperref}
\PassOptionsToPackage{hyphens}{url}
%
\documentclass[
]{article}
\usepackage{amsmath,amssymb}
\usepackage{lmodern}
\usepackage{ifxetex,ifluatex}
\ifnum 0\ifxetex 1\fi\ifluatex 1\fi=0 % if pdftex
  \usepackage[T1]{fontenc}
  \usepackage[utf8]{inputenc}
  \usepackage{textcomp} % provide euro and other symbols
\else % if luatex or xetex
  \usepackage{unicode-math}
  \defaultfontfeatures{Scale=MatchLowercase}
  \defaultfontfeatures[\rmfamily]{Ligatures=TeX,Scale=1}
\fi
% Use upquote if available, for straight quotes in verbatim environments
\IfFileExists{upquote.sty}{\usepackage{upquote}}{}
\IfFileExists{microtype.sty}{% use microtype if available
  \usepackage[]{microtype}
  \UseMicrotypeSet[protrusion]{basicmath} % disable protrusion for tt fonts
}{}
\makeatletter
\@ifundefined{KOMAClassName}{% if non-KOMA class
  \IfFileExists{parskip.sty}{%
    \usepackage{parskip}
  }{% else
    \setlength{\parindent}{0pt}
    \setlength{\parskip}{6pt plus 2pt minus 1pt}}
}{% if KOMA class
  \KOMAoptions{parskip=half}}
\makeatother
\usepackage{xcolor}
\IfFileExists{xurl.sty}{\usepackage{xurl}}{} % add URL line breaks if available
\IfFileExists{bookmark.sty}{\usepackage{bookmark}}{\usepackage{hyperref}}
\hypersetup{
  pdftitle={MATH50003 Week4},
  hidelinks,
  pdfcreator={LaTeX via pandoc}}
\urlstyle{same} % disable monospaced font for URLs
\usepackage[margin=1in]{geometry}
\usepackage{graphicx}
\makeatletter
\def\maxwidth{\ifdim\Gin@nat@width>\linewidth\linewidth\else\Gin@nat@width\fi}
\def\maxheight{\ifdim\Gin@nat@height>\textheight\textheight\else\Gin@nat@height\fi}
\makeatother
% Scale images if necessary, so that they will not overflow the page
% margins by default, and it is still possible to overwrite the defaults
% using explicit options in \includegraphics[width, height, ...]{}
\setkeys{Gin}{width=\maxwidth,height=\maxheight,keepaspectratio}
% Set default figure placement to htbp
\makeatletter
\def\fps@figure{htbp}
\makeatother
\setlength{\emergencystretch}{3em} % prevent overfull lines
\providecommand{\tightlist}{%
  \setlength{\itemsep}{0pt}\setlength{\parskip}{0pt}}
\setcounter{secnumdepth}{-\maxdimen} % remove section numbering
\ifluatex
  \usepackage{selnolig}  % disable illegal ligatures
\fi

\title{MATH50003 Week4}
\author{}
\date{\vspace{-2.5em}}

\begin{document}
\maketitle

\hypertarget{week-4-decomposition-and-least-sqs.}{%
\section{\texorpdfstring{\textbf{Week 4 Decomposition and Least
Sqs.}}{Week 4 Decomposition and Least Sqs.}}\label{week-4-decomposition-and-least-sqs.}}

Importance of decomposition due to the ease of invertibility, hence
quicker solution to linear systems.

\hypertarget{qr}{%
\subsubsection{\texorpdfstring{\textbf{QR}}{QR}}\label{qr}}

\(A=QR\) \(Q\) orthogonal, \(R\) right triangular (rectangular), last
few rows of zeros

\[  
\qquad A = Q R = \underbrace{\begin{bmatrix} 𝐪_1 | \cdots | 𝐪_m \end{bmatrix}}_{m × m}
\underbrace{\begin{bmatrix} × & \cdots & × \\ 
& ⋱ & ⋮ \\ 
&& × \\ 
&&0 \\ 
&&⋮ \\ 
&& 0 
\end{bmatrix}}_{m × n}
\]

\hypertarget{reduced-qr}{%
\subsubsection{\texorpdfstring{\textbf{Reduced
QR}}{Reduced QR}}\label{reduced-qr}}

\begin{itemize}
\item
  Drop the last few rows of zeros , the right triangular is now square
  and triangular. Also \(Q\) becomes \(\hat{Q}\) correspondingly
\item
  If we take transpose and then decompose, we get kernel as the
  orthogonal space to row space is column kernel
\end{itemize}

\hypertarget{least-squares}{%
\subsubsection{\texorpdfstring{\textbf{Least
Squares}}{Least Squares}}\label{least-squares}}

Mnimize \[
||A \vec{x} - \vec{b}||
\] - If square matrix and invertible then easily solved by inverse. -
Using \textbf{QR}:

\[
||QR\vec{x}-\vec{b}||=||Q^T(QR\vec{x}-\vec{b})||=||R\vec{x} - Q^T b||
\]

As orthogonal matrices keep norm.\\
- So drop the zeros and minimize \[||\hat{R}\vec{x} - \hat{Q}^T b||\]

\begin{itemize}
\tightlist
\item
  if \(A\) has full rank, then \(\hat{R}\) is invertible (since
  orthogonal \(Q\) has full rank, so dim of column span of rectangular
  \(R\) is full, hence \(\hat{R}\) invertible), so
\end{itemize}

\[x=\hat{R}^{-1}\hat{Q}^Tb\]

\hypertarget{computing-qr-via-gram-schmidt}{%
\subsubsection{\texorpdfstring{\textbf{Computing QR via
Gram-Schmidt}}{Computing QR via Gram-Schmidt}}\label{computing-qr-via-gram-schmidt}}

\begin{itemize}
\item
  Since \(R\) is upper-triangular, the span of first \(k\) columns of
  \(Q\) is the same as \(A\).
\item
  First compute the `coefficients' of G-S namely the dot product of
  already computed orthonormal vectors \(q_k\) with the to be
  transformed basis vector \(a_j\), store those \(q_k^Ta_j\) as
  \(r_{k,j}\) in \(R\).
\end{itemize}

\[𝐯_j := 𝐚_j - \sum_{k=1}^{j-1} \underbrace{𝐪_k^\top 𝐚_j}_{r_{kj}}𝐪_k\]

\begin{itemize}
\item
  The diagonal entries \(r_{j,j}\) are the norm of \(v_j\), where
  \[q_j=\frac{v_j}{||v_j||}\]
\item
  Common G-S works over the columns of \(R\) and fill each of them, when
  filling the \(r_{k,j}\)'s entries, we take \(O(m)\), other norm
  computing stuff takes \(O(1)\), so each column takes \(O(mj)\),
  summing up gives \(O(mn^2)\)
\end{itemize}

\hypertarget{computing-qr-via-householder}{%
\subsubsection{\texorpdfstring{\textbf{Computing QR via
Householder}}{Computing QR via Householder}}\label{computing-qr-via-householder}}

More numerically stable than \textbf{G-S}.

\begin{itemize}
\item
  Repeatedly apply the reflection matrix and use the property of
  Householder
  \[Q_{\mathbf{x}}\mathbf{x}=\pm ||\mathbf{x}||\mathbf{e_1}\] to only
  keep the first entry
\item
  After altering the first column, use modified second column as the
  \(\mathbf{x}\) in Householder and apply to the sub-matrix (keeping
  first row and column fixed)
\item
  Continue inductively
\end{itemize}

\$\$Q\_2 Q\_1A =

\begin{bmatrix} 
\times & \times & \times & \cdots & \times \\

& \times & \times & \cdots & \times \\

 && ⋮ & ⋱ & ⋮ \\

 && \times & \cdots & \times \end{bmatrix}

\$\$

\begin{itemize}
\tightlist
\item
  So final \(Q = Q_1 \cdots Q_n = (Q_n \cdots Q_1)^T\)
\end{itemize}

\hypertarget{lu-factorisation}{%
\subsubsection{\texorpdfstring{\textbf{LU
factorisation}}{LU factorisation}}\label{lu-factorisation}}

\begin{itemize}
\item
  Repeatedly apply lower triangular matrices with one columns only to
  \(A\)
\item
  Analogously as above, inductively apply to sub-matrices that are
  modified on the previous step; the lower triangular matrix has its
  columns `move down' in the process
\end{itemize}

\$\$L\_1 =

\begin{bmatrix} 1 \\ -{a_{21} \over a_{11}} & 1 \\ ⋮ &&⋱ \\

 -{a_{n1} \over a_{11}} &&& 1

\end{bmatrix}

\$\$

\[L_2 = \begin{bmatrix} 1 \\ 
 & 1 \\ 
& -\frac{a_{32}^1}{a_{22}^1} & 1 \\
& ⋮ & & \ddots \\
& -\frac{a_{n2}^1}{a_{22}^1} & \cdots && 1
\end{bmatrix}\]

\begin{itemize}
\tightlist
\item
  Now \(L_n \cdots L_1 A = U\), and note by the properties of lower
  triangular matrices, can first move \(L_k\) to RHS and invert by the
  fact (inverse is the negative column) \[
  L_j^{-1}  = I - \begin{bmatrix} 𝟎_j \\ 𝐥_j \end{bmatrix} 𝐞_j^⊤
  \]
\item
  Also note the product of lower triangular is lower triangular.
\end{itemize}

\hypertarget{plu-decomposition}{%
\subsubsection{\texorpdfstring{\textbf{PLU
decomposition}}{PLU decomposition}}\label{plu-decomposition}}

\begin{itemize}
\item
  Apply permutation matrix to permute the largest entry to the first
  before \textbf{LU}.
\item
  When getting back, use property that
  \(P_{\sigma} L_j = \tilde{L_j} P_{\sigma}\), with ,(\(\sigma\) only
  changing the first \(j\) entries) , the \(\mathbf{l_j}\) above
  permuted by the last \(j+1\)permutation of \(P_{\sigma}\).
\end{itemize}

\hypertarget{cholesky}{%
\subsubsection{\texorpdfstring{\textbf{Cholesky}}{Cholesky}}\label{cholesky}}

\begin{itemize}
\tightlist
\item
  The diagonal entries of a positive definite matrix are positive
  (simply take \(x=e_k\), \(x^T A x >0\)) \[
  A = \begin{bmatrix} α & 𝐯^\top \\
                    𝐯   & K
                    \end{bmatrix} = \underbrace{\begin{bmatrix} \sqrt{α} \\ 
                                    {𝐯 \over \sqrt{α}} & I \end{bmatrix}}_{L_1}
                                    \underbrace{\begin{bmatrix} 1  \\ & K - {𝐯 𝐯^\top \over α} \end{bmatrix}}_{A_1}
                                    \underbrace{\begin{bmatrix} \sqrt{α} & {𝐯^\top \over \sqrt{α}} \\
                                     & I \end{bmatrix}}_{L_1^\top}.
  \]\\
\item
  Induct down by the fact that subslices of positiv def.
\end{itemize}

\hypertarget{speed-and-stability}{%
\subsubsection{\texorpdfstring{\textbf{Speed and
Stability}}{Speed and Stability}}\label{speed-and-stability}}

\begin{itemize}
\item
  \textbf{Cholesky} \(>\) \textbf{LU} \(>\) \textbf{QR} in terms of
  speed
\item
  \textbf{Cholesky} and \textbf{QR} (Householder) \(>\) \textbf{LU},
  \textbf{PLU} is in theory unstable but normally stable
\end{itemize}

\end{document}
