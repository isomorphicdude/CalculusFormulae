\documentclass{article}
\usepackage[utf8]{inputenc}
\usepackage[utf8]{inputenc}
\usepackage{amsfonts, amsmath, amssymb, 
            hyperref, float, geometry,
             amsthm, setspace} % add use svg when needed
            
% \newgeometry{vmargin={15mm}, hmargin={12mm,12mm}}
% \svgpath{{figures/}}

\newcommand{\R}{\mathbb{R}}
\newcommand{\N}{\mathbb{N}}
\newcommand{\mf}[1]{\mathbf{#1}}
\newcommand{\ca}[1]{\mathcal{#1}}

% for algorithms
\makeatletter
\newcommand{\HEADER}[1]{\ALC@it\underline{\textsc{#1}}\begin{ALC@g}}
\newcommand{\ENDHEADER}{\end{ALC@g}}
\makeatother

% for numbering the theorems            
\theoremstyle{plain}
\newtheorem{thm}{Theorem}[section]
\newtheorem{lemma}[thm]{Lemma}
\newtheorem{cor}[thm]{Corollary}
%%%%%%%%%%%%%%%%%%%%%%%%%%%%%%%%%%%%%%%%%%%%%%%%%%%%%%%%%%%
% the following are not in italics
\theoremstyle{definition}
\newtheorem{defn}[thm]{Definition}
\newtheorem{eg}[thm]{Example}
\newtheorem{remark}[thm]{Remark}
\newtheorem{prop}[thm]{Proposition}

%%%%%%%%%%%%%%%%%%%%%%%%%%%%%%%%%%%%%%%%%%%%%%%%%%%%%%%%%%%%
\doublespacing

\setlength\parindent{0pt}

\title{Functional Analysis}
\author{Typed by: tw1320@ic.ac.uk}
\date{\today}

\begin{document}

\maketitle

\section{Preliminaries}   

\subsection{Norms and Metrics}

\begin{defn}
(Metric) Let $X$ be a nonempty set. 
A function $d: X \times X \to \R^+$ satisfying the following is called a  \textbf{metric}  

\begin{itemize}
    \item (Positive definitiness) $\forall x,y \in X, d(x,y)\geq 0$ if $x \neq y$ and $d(x,y)=0 \iff x=y$
    \item (Symmetry) $\forall x,y \in X, d(x,y)=d(y,x)$
    \item (Triangle-inequality) $\forall x,y,z \in X, d(x,y) \leq d(x,z) + d(z,y)$
\end{itemize}

\end{defn}
\begin{defn}
    (Translation invariant) A metric $d$ is \textbf{translation invariant} if $\forall x,y \in X, d(x,y)=d(x+a,y+a)$ for all $a \in X$.
\end{defn} 

\begin{eg}
    The Euclidean metric on $\R^n$ is translation invariant.  
    But the metric $d(x,y)=|x^3-y^3|$ on $\R$ is not translation invariant.
\end{eg}

To introduce the idea of a metric linear space, we need to define metrics on product spaces.  

\begin{defn} (Metric on Product Spaces)
    Given metric $\rho$ on a vector space $V$ over $\mathbb{K}$, a metric on $V \times V$ is defined by:  
    \begin{equation*}
        d((a,b),(c,d)) = (\rho(a,c)^p+\rho(b,d)^p)^{1/p}, p \in [1,\infty)
    \end{equation*}  
    and on $\mathbb{K} \times V$ by:
    \begin{equation*}
        d((\lambda, a),(\lambda',a')) = \max \{|\lambda-\lambda'|, \rho (a, a')\}
    \end{equation*}
\end{defn}

\begin{defn}
    (Metric Linear Spaces) 
    A pair $(X, d)$ with $X$ being a linear space over $\mathbb{K}$ and $d$ being a metric is called a \textbf{metric linear space} if and only if
    addition and multiplication by scalar are continuous.
\end{defn}   


In other words, the following are true:  

\begin{itemize}
    \item $x_n \to x, \quad y_n \to y \implies x_n + y_n \to x + y$
    \item $\lambda_n \to \lambda, \lambda_n, \lambda \in \mathbb{K},  x_n \to x \implies \lambda_n x_n \to \lambda x$
\end{itemize}  

It is easily verified and if $d$ is translation invariant, then addition of vectors is continuous: namely,
$d(x_n+y_n, x+y)=d(x_n-x, y-y_n) \leq d(x_n-x,0)+ d(y_n-y, 0)$.  However, a translation invariant metric does not
guarantee that multiplication by scalar is continuous.  

\begin{eg}
    Let $X$ be the space of all sequences in $\R$ and $d(x,y) = \sup_{i\in \N} |x^i-y^i|^{1/i}$, where the $x^i$
    denotes the $i^{th}$ element of the sequence $x$. Then $d$ is a metric on $X$ and it is translation invariant.  
    
    Take $(x^i_n)_{i\in \N}=(a)_{i \in \N}$, a constant sequence with $a>1$, and a scalar $\lambda_n = \xi^n, \xi \in (0,1)$, so that $\lambda_n \to 0$
    and $\lambda_n x_n \to 0$.  
    \begin{equation*}
        d(\lambda_n x_n, 0) = \sup_{i\in \N} |\xi|^{n/i} |a|^{1/i} \geq 1
    \end{equation*}  

    So multiplication by scalar is not continuous.
\end{eg}

\begin{defn}
    (Norm) Let $X$ be a nonempty set. A function $||\cdot||: X \to \R^+$ satisfying the following 
    is called a \textbf{norm}:
    \begin{itemize}
        \item (Positive definitiness) $\forall x \in X, ||x|| \geq 0$ and $||x||=0 \iff x=0$
        \item (Triangle-inequality) $\forall x,y \in X, ||x+y|| \leq ||x|| + ||y||$
        \item (Homogeneity) $\forall x \in X, \forall \lambda \in \mathbb{K}, ||\lambda x|| = |\lambda| ||x||$
    \end{itemize}
\end{defn}  

\begin{remark}
    Norm is a continuous function.
\end{remark}

\begin{defn}
(Normed Linear Spaces) A pair $(X, ||\cdot||)$ with $X$ being a linear space over $\mathbb{K}$ and 
                    $||\cdot||$ being a norm is called a \textbf{normed linear space}
\end{defn} 

Note that every normed linear space is a metric space, since every norm can induce a metric by $d(x,y)=||x-y||$. 
However, not every metric is a norm.  

\begin{eg}
    Let $X$ be the space of all sequences in $\R$ and $z>1$. A translation invariant metric $d$ is defined by  
    \begin{equation*}
        d(x,y) = \sum_{i=1}^{\infty} z^{-n} \dfrac{|x_i-y_i|}{1+|x_i-y_i|}
    \end{equation*}  
    But $d$ is not a norm, as it is not homogenous.
\end{eg}  

\begin{eg}
    Let $X=\R$ be the real numbers and $|\cdot|$ the Euclidean norm. Another example of a metric that is not a norm 
    is given by:
    \begin{equation*}
        d(x,y) = \min \{|x-y|, 1\}
    \end{equation*}  
    this is not a norm because it is not homogenous. (Note also that it is not translation invariant.)
\end{eg}  
%%%%%%%%%%%%%%%%%%%%%%%%%%%%%%%%%%%%%%%%%%%%%%%%%%%%%%%%%%%%%%%
% move this section to later%%%%%%%%%%%%%%%%%%%%%%%%%%%%%%%%%%%%%%%%%%%%%%
\subsection{Common Spaces}  

\paragraph*{$l_p$ Spaces}    
For $p \in [1,\infty)$, the space $l_p$ is defined as the set of all sequences $(x_n)_{n\in \N}$ such that  

\begin{equation*}
    \sum_{n=1}^{\infty} |x_n|^p < \infty
\end{equation*}

the function  

\begin{equation*}
    ||x||_p = \left(\sum_{i=1}^n |x_i|^p\right)^{1/p}
\end{equation*}  

defines a norm on $l_p$.  
\begin{remark}
    $l_p \subset l_q$ when $p < q$. And $\lim_{p\to \infty} ||x||_p = ||x||_{\infty}$
\end{remark}
\paragraph*{$l_\infty$ Spaces}  
The space $l_\infty$ is defined as the set of all sequences $(x_n)_{n\in \N}$ such that

\begin{equation*}
    \sup_{n\in \N} |x_n| < \infty
\end{equation*}

the function

\begin{equation*}
    ||x||_\infty = \sup_{n\in \N} |x_n|
\end{equation*}

defines a norm on $l_\infty$.

%%%%%%%%%%%%%%%%%%%%%%%%%%%%%%%%%%%%%%%%%%%%%%%%%%%%%%%%%%%%%%%
%%%%%%%%%%%%%%%%%%%%%%%%%%%%%%%%%%%%%%%%%%%%%%%%%%%%%%%%%%%%%%%

\subsection{Inequalities}

% Young's inequality
\begin{prop}
\label{young}
(Young) If $p>1$ and $q$ is defined by $\frac{1}{p}+\frac{1}{q}=1$ (such $p,q$ are called conjugates), then for $a,b \geq 0$  
\begin{equation}
    a^{\frac{1}{p}}b^{\frac{1}{q}} \leq \frac{a}{p} + \frac{b}{q} 
\end{equation}
\end{prop}  

\begin{proof}
    (Sketch) Consider the function $f(t) = t^{\alpha}-\alpha t+\alpha-1$, where $\alpha \in (0,1), t\geq 0$,
    $f(1)=0$ is a maximum and consider $f(\frac{a}{b})\leq 0$ with $\alpha=\frac{1}{p}$.  
\end{proof}

% Holder and Minkowski inequalities
\begin{cor}
    The following inequalities are results of \autoref{young}:  
    \begin{itemize}
        \item (Hölder) If $p,q$ are conjugates, then for complex numbers $x_1, \dots, x_n$ and $y_1,\dots, y_n$:    
                        \begin{equation}
                            \sum_{i=1}^n |x_i y_i| \leq \left[\sum_{i=1}^n |x_i|^p\right]^{1/p} \left[\sum_{i=1}^n |y_i|^q\right]^{1/q}
                        \end{equation}  

                    For ${x_i} \in l_p, {y_i} \in l_q$, then:  
                    \begin{equation}
                        \sum_{i=1}^{\infty} |x_i y_i| \leq \left[\sum_{i=1}^{\infty} |x_i|^p\right]^{1/p} \left[\sum_{i=1}^{\infty} |y_i|^q\right]^{1/q}
                    \end{equation}  
                    When $p=q=2$, this is the Cauchy-Schwarz inequality.  

                    For functions $f \in L^p, g \in L^q$, then:  
                        \begin{equation}
                            f \cdot g \in L^1 \quad \text{and} \quad ||fg||_{L^1} \leq ||f||_{L^p} ||g||_{L^q}
                        \end{equation}
        \item (Minkowski) If $p \geq 1$, then for complex numbers  $x_1, \dots, x_n$ and $y_1,\dots, y_n$:
                        \begin{equation}
                            \left[\sum_{i=1}^n |x_i+y_i|^p \right]^{1/p} \leq \left[\sum_{i=1}^n |x_i|^p\right]^{1/p} + \left[\sum_{i=1}^n |y_i|^p\right]^{1/p}
                        \end{equation}
                    For ${x_i}, {y_i} \in l_p$, then:
                    \begin{equation}
                        \left[\sum_{i=1}^{\infty} |x_i+y_i|^p \right]^{1/p} \leq \left[\sum_{i=1}^{\infty} |x_i|^p\right]^{1/p} + \left[\sum_{i=1}^{\infty} |y_i|^p\right]^{1/p}
                    \end{equation}
                    For functions $f, g \in L^p$, then:
                        \begin{equation}
                            f+g \in L^p \quad \text{and} \quad ||f+g||_{L^p} \leq ||f||_{L^p} + ||g||_{L^p}
                        \end{equation}
    \end{itemize}
    \begin{proof}
        (Sketch) For the Hölder inequality, use Young's inequality with $a=(\dfrac{|x_i|}{||\mf{x}||_p})^p$ and 
        $b=(\dfrac{|y_i|}{||\mf{y}||_q})^q$ (use $L^p$ norm when proving for functions).  
        
        For Minkowski, use $(|x_i+y_i|^{p-1})(|x_i|+|y_i|)$ to break down the LHS, then use Hölder's inequality,
        $\sum_{i=1}^n (|x_i+y_i|^{p-1})|x_i| \leq \left[ \sum_{i=1}^n |x_i|^p\right]^{1/p}
        \left[ \sum_{i=1}^n ((|x_i|+|y_i|)^{p-1})^q\right]^{1/q}$ and sum up the inequalities.   
        For the $l^p$ case, first note that $p=1,\infty$ cases are obvious, then note
        that $|x_i+y_i|^{p/q}$ is in $l^q$ and use Hölder's inequality as before on 
        $\sum_{i=1}^{\infty} (|x_i+y_i|^{p/q})|x_i|+\sum_{i=1}^{\infty} (|x_i+y_i|^{p/q})|y_i|$.  
        The $L^p$ case is similar.  
    \end{proof}
\end{cor}

\begin{defn}
    (Convex functions) A function $f$ is convex if 
    \begin{equation*}
        f(\alpha x + (1-\alpha)y) \leq \alpha f(x) + (1-\alpha)f(y)
    \end{equation*}
    $\forall x,y \in V$ and $\forall \alpha \in [0,1]$.
\end{defn}

Concave functions are defined similarly but with the inequality reversed. We also note that all convex functions defined
on an open interval is \textbf{continuous} on that interval but not every convex function is continuous. An example being
$f(x) = - \sqrt{x}, x>0$ and $f(0)=1$; it is convex on $[0,1)$ but clearly not continuous at $0$.

\begin{prop}
    (Equivalent forms of convexity) If $f: I \to \R$ is a twice differentiable function,  
    \begin{itemize}
        \item If $f''(x)\geq 0, \forall x \in I$.
        \item If $\forall y \in I$, there exists $\gamma \in \R$, such that $\forall x \in I$, 
                $\gamma (x-y) \leq f(x)-f(y)$.
    \end{itemize}
\end{prop}


\begin{prop}
    (Triangle inequality for concave functions)  
    If $f: \R_+ \to \R_+$ is concave and $f(0)=0$, then for $x,y \in \R_+$:
    \begin{equation*}
        f(x+y) \leq f(x) + f(y)
    \end{equation*}
\end{prop}

The above proposition is useful when considering different norms on $\R$. For instance, the function
$f(x)=x^p$, for $p\in (0,1)$.  

\begin{prop}
(Jensen) 
For real continuous convex function $f$ and positive weights satisfying $\sum_{i=1}^n \alpha_i=1$,  

\begin{equation*}
    f\left(\sum_{i=1}^n \alpha_i x_i\right) \leq \sum_{i=1}^n \alpha_i f(x_i)
\end{equation*}

If the function is concave, then the inequality is reversed.  
The equality is attained when $x_i's$ are equal or $f$ is linear.
\end{prop}


\section{Completeness and Separability}  

\begin{defn}
    (Completeness)
    A metric space in which every Cauchy sequence converges to some limit in that space is called 
    \textbf{complete}.  
\end{defn}  

\begin{defn}
    (Banach Space) 
    A \textbf{Banach Space} is a complete normed vector space.
\end{defn}

\begin{thm}
    (Completion)  
    For every metric space $(X, d)$, there exists a complete metric space $(\tilde{X}, \tilde{d})$ 
    and an isometry $i: X \to \tilde{X}$.   
\end{thm}  

\begin{remark}
    The metric space $X$ is isometric to a dense subset of its completion $\tilde{X}$.  
    We construct the completeion by introducing equivalence relation of Cauchy sequences.
\end{remark}

\begin{defn}
    (Separability) 
    A metric space is \textbf{separable} if it has a countable dense subset.  
\end{defn}  

We now present some results of completeness and separability of a few common spaces.    

\subsection{Common Spaces}  

\paragraph*{$l_p$ Spaces}    
For $p \in [1,\infty)$, the space $l_p$ is defined as the set of all sequences $(x_n)_{n\in \N}$ such that  

\begin{equation*}
    \sum_{n=1}^{\infty} |x_n|^p < \infty
\end{equation*}

the function  

\begin{equation*}
    ||x||_p = \left(\sum_{i=1}^n |x_i|^p\right)^{1/p}
\end{equation*}  

defines a norm on $l_p$.  
\begin{remark}
    $l_p \subset l_q$ when $p < q$. And $\lim_{p\to \infty} ||x||_p = ||x||_{\infty}$
\end{remark}  

\begin{prop}
    $l_p$ is a Banach space for $p \in [1,\infty)$.
\end{prop}  

\begin{prop}
    $l_p$ is separable for $p \in [1,\infty)$.
\end{prop}  


\paragraph*{$l_\infty$ Spaces}  
The space $l_\infty$ is defined as the set of all sequences $(x_n)_{n\in \N}$ such that

\begin{equation*}
    \sup_{n\in \N} |x_n| < \infty
\end{equation*}

the function

\begin{equation*}
    ||x||_\infty = \sup_{n\in \N} |x_n|
\end{equation*}

defines a norm on $l_\infty$.  

\begin{prop}
    $l_\infty$ is a Banach space.
\end{prop}

\begin{prop}
    $l_{\infty}$ is \textbf{not} separable.  
\end{prop} 

\paragraph*{$L^p$ spaces}  

See measure theory notes (I will fill it in later).

\paragraph*{$C[a,b]$ Spaces}  
The space of continuous functions on $[a,b]$ equipped with the supremum norm is denoted by $C[a,b]$.  
The norm is defined by   

\begin{equation*}
    ||f||_{\infty}=\sup_{x \in [a,b]} |f(x)|
\end{equation*}

\begin{prop}
    $C[a,b]$ is a Banach space.
\end{prop}  

\begin{thm}
    $C[a,b]$ is separable.
\end{thm}  

\begin{lemma}
    
\end{lemma}  

\begin{proof}

\end{proof}

\paragraph*{Special sequence spaces $c, c_0, c_{00}$}  

\begin{itemize}
    \item $c$ is the set of all convergent sequences in $l_{\infty}$
    \item $c_0$ is the set of all sequences converging to zero in $l_{\infty}$
    \item $c_{00}$ is the set of all sequences such that $x_n=0$ for all
          but finitely many $n$ in $l_{\infty}$
\end{itemize}  

Note that they are all equipped with the supremum norm.  

\begin{remark}
    Note that $c_{00} \subset c_{0} \subset c$
\end{remark}  

We discuss their completeness and separability.  

\begin{prop}
    $c$ and $c_0$ are both Banach spaces.
\end{prop}  

\begin{eg}
    Note that the space $c_{00}$ is not a complete space, as it is not
    closed in $c$; take the sequence $x_n = (1, \frac{1}{2}, \ldots, \frac{1}{n}, 0, \ldots)$
\end{eg}

\begin{prop}
    $c$ and $c_0$ are separable.
\end{prop}  


\paragraph*{Sequence space $s$}  
(not in the notes but can fill it in later)  

\section{Infinite Dimensional Spaces}  

There are some fundamental differences between finite and infinite dimensional spaces.  

\begin{defn}
    (Equivalent Norms)
    Two norms $||\cdot||_1, ||\cdot_2||$ on a vector space $X$ are \textbf{equivalent} 
    if there exists a constant $C \in [1, \infty)$ such that  
    \begin{equation*}
        \forall x \in X, \quad \frac{1}{C}||x||_1 \leq ||x||_2 \leq C||x||_1
    \end{equation*}
\end{defn}  

\subsection{Properties of Finite Dimensional Spaces}

\begin{thm}
    In finite dimensional spaces, all norms are equivalent.
\end{thm}  

However, this is not true in infinite dimensional spaces.  

\begin{eg}
    Consider $f_n(t)=t^n$ in $C[0,1]$. This sequence of functions converge to $0$ 
    in $||\cdot||_1$ but not in $||\cdot||_{\infty}$.
\end{eg}

\begin{prop}
    All finite dimensional spaces are complete; hence all finite dimensional spaces are closed.
\end{prop}  

\subsection{Compactness}
\begin{defn}
    (Compact) 
    A set $K \subset X$ is \textbf{compact} 
    if every sequence in $K$ has a convergent subsequence with limit in $K$.
\end{defn}  

\begin{remark}
    A set $K$ is compact if and only if it is bounded and closed in finite dimensional spaces.  
    Compactness implies closed and boundedness even in infinite dimensional spaces but the 
    other direction is not true.
\end{remark}  

\begin{eg}
    Take the following set in $l^1$:  
    \begin{equation*}
        K = \left\{ e_n = (0, \ldots, 0, 1, 0, \ldots), n \in \mathbb{N} \right\}
    \end{equation*}  
    This set is bounded and closed but clearly not compact as each term in the sequence
    are distance $2$ apart.  
\end{eg}  

Another example is the following  

\begin{eg}
    Consider the set $\bar{B}_1 \subset C[0,1]$,  
    \begin{equation*}
        \bar{B}_1 = \left\{ f \in C[0,1] : ||f||_{\infty} \leq 1 \right\}
    \end{equation*}  
    The sequence of functions,  
    \begin{equation*}
        f_n(t) = \sin(2^n \pi t), \quad 0 \leq t \leq 1
    \end{equation*}  
    Then $||f_n-f_m||\geq 1$ for all $n \neq m$ and hence no convergent subsequences. 
\end{eg}

\begin{thm}
    In a normed space $(X, ||\cdot||)$, the following two conditions are equivalent:  
    \begin{itemize}
        \item $\dim X < \infty$
        \item The unit ball $\bar{B}_1$ is compact 
    \end{itemize}
\end{thm}  

Before proving this, we need a lemma that allows us to construct a sequence of vectors
that are at a fixed distance to each other, hence has no convergent subsequence.  

\begin{lemma}
    (Riesz)
\end{lemma}

\subsection{Basis in Infinite Dimensional Spaces}  

Similar to finite dimensional vector spaces, there is something "analogous" to 
a basis in infinite dimensional spaces.  

\begin{defn}
    (Hamel)
\end{defn}  

\begin{defn}
    (Schauder)
\end{defn}   

\begin{eg}
    Schauder basis of $l^p$ is the set of all sequences of the form $e_n$ but there is no 
    Schauder basis for $l^{\infty}$.  
\end{eg}  

Sometimes the Schauder basis might look complicated.    
\begin{eg}
    A Schauder basis of $C[0,1]$ is
\end{eg}  

We also provide some counter examples of Schauder bases in $C[0,1]$.  

\begin{eg}

\end{eg}

\begin{thm}
    If $(X, ||\cdot||)$ has a Schauder basis, then $X$ is separable.  
\end{thm}  



\section{Hilbert Spaces}  

\begin{defn}
    (Inner Product)
\end{defn}  

\begin{remark}
    The inner product is continuous.  
\end{remark}

\begin{defn}
    (Hilbert Space)
\end{defn}  

The inner product and norm naturally leads to the famous inequality.  

\begin{prop}
    (Cauchy-Schwarz)
\end{prop}

\begin{eg}
    $l^2$ is a Hilbert space.
\end{eg} 

\begin{prop}
    (Important Identities)
    \begin{itemize}
        \item (Parallelogram)
        \item (Polarization)
    \end{itemize}
\end{prop}  

\begin{prop}
    $l^p$ is not a Hilbert space for $p \neq 2$.
\end{prop}  

\subsection{Orthogonality}  

\begin{defn}
    (Orthogonal)
\end{defn}  

\begin{thm}
    (Projection)
\end{thm}  

\begin{thm}
    (Orthogonal Complement)
\end{thm}

\begin{thm}
    (Riesz Representation)
\end{thm}


\section{Linear Operators}  

\section{Dual Spaces}  

\section{The Hahn Banach Theorems}  

\section{The Uniform Boundedness Theorem}
\subsection{Baire's Category Theorem}
\section{The Open Mapping Theorem}  

\section{The Closed Graph Theorem}  

\section{Compact Operators}
\end{document}
