\documentclass{article}
\usepackage[utf8]{inputenc}
\usepackage[utf8]{inputenc}
\usepackage{amsfonts, amsmath, amssymb, 
            hyperref, float, geometry,
             amsthm, setspace} % add use svg when needed
            
% \newgeometry{vmargin={15mm}, hmargin={12mm,12mm}}
% \svgpath{{figures/}}

\newcommand{\R}{\mathbb{R}}
\newcommand{\N}{\mathbb{N}}
\newcommand{\mf}[1]{\mathbf{#1}}
\newcommand{\ca}[1]{\mathcal{#1}}

% for algorithms
\makeatletter
\newcommand{\HEADER}[1]{\ALC@it\underline{\textsc{#1}}\begin{ALC@g}}
\newcommand{\ENDHEADER}{\end{ALC@g}}
\makeatother

% for numbering the theorems            
\theoremstyle{plain}
\newtheorem{thm}{Theorem}[section]
\newtheorem{lemma}[thm]{Lemma}
\newtheorem{cor}[thm]{Corollary}
%%%%%%%%%%%%%%%%%%%%%%%%%%%%%%%%%%%%%%%%%%%%%%%%%%%%%%%%%%%
% the following are not in italics
\theoremstyle{definition}
\newtheorem{defn}[thm]{Definition}
\newtheorem{eg}[thm]{Example}
\newtheorem{remark}[thm]{Remark}
\newtheorem{prop}[thm]{Proposition}

%%%%%%%%%%%%%%%%%%%%%%%%%%%%%%%%%%%%%%%%%%%%%%%%%%%%%%%%%%%%
\doublespacing

\setlength\parindent{0pt}

\title{Functional Analysis}
\author{Typed by: tw1320@ic.ac.uk}
\date{\today}

\begin{document}

\maketitle

\section{Preliminaries}   

\subsection{Norms and Metrics}

\begin{defn}
(Metric) Let $X$ be a nonempty set. 
A function $d: X \times X \to \R^+$ satisfying the following is called a  \textbf{metric}  

\begin{itemize}
    \item (Positive definitiness) $\forall x,y \in X, d(x,y)\geq 0$ if $x \neq y$ and $d(x,y)=0 \iff x=y$
    \item (Symmetry) $\forall x,y \in X, d(x,y)=d(y,x)$
    \item (Triangle-inequality) $\forall x,y,z \in X, d(x,y) \leq d(x,z) + d(z,y)$
\end{itemize}

\end{defn}
\begin{defn}
    (Translation invariant) A metric $d$ is \textbf{translation invariant} if $\forall x,y \in X, d(x,y)=d(x+a,y+a)$ for all $a \in X$.
\end{defn} 

To introduce the idea of a metric linear space, we need to define metrics on product spaces.  

\begin{defn} (Metric on Product Spaces)
    Given metric $\rho$ on a vector space $V$ over $\mathbb{K}$, a metric on $V \times V$ is defined by:  
    \begin{equation*}
        d((a,b),(c,d)) = (\rho(a,c)^p+\rho(b,d)^p)^{1/p}, p \in [1,\infty)
    \end{equation*}  
    and on $\mathbb{K} \times V$ by:
    \begin{equation*}
        d((\lambda, a),(\lambda',a')) = \max \{|\lambda-\lambda'|, \rho (a, a')\}
    \end{equation*}
\end{defn}

\begin{defn}
    (Metric Linear Spaces) 
    A pair $(X, d)$ with $X$ being a linear space over $\mathbb{K}$ and $d$ being a metric is called a \textbf{metric linear space} if and only if
    addition and multiplication by scalar are continuous.
\end{defn}   


In other words, the following are true:  

\begin{itemize}
    \item $x_n \to x, \quad y_n \to y \implies x_n + y_n \to x + y$
    \item $\lambda_n \to \lambda, \lambda_n, \lambda \in \mathbb{K},  x_n \to x \implies \lambda_n x_n \to \lambda x$
\end{itemize}  

It is easily verified and if $d$ is translation invariant, then addition of vectors is continuous: namely,
$d(x_n+y_n, x+y)=d(x_n-x, y-y_n) \leq d(x_n-x,0)+ d(y_n-y, 0)$.  However, a translation invariant metric does not
guarantee that multiplication by scalar is continuous.  

\begin{eg}
    Let $X$ be the space of all sequences in $\R$ and $d(x,y) = \sup_{i\in \N} |x^i-y^i|^{1/i}$, where the $x^i$
    denotes the $i^{th}$ element of the sequence $x$. Then $d$ is a metric on $X$ and it is translation invariant.  
    
    Take $(x^i_n)_{i\in \N}=(a)_{i \in \N}$, a constant sequence with $a>1$, and a scalar $\lambda_n = \xi^n, \xi \in (0,1)$, so that $\lambda_n \to 0$
    and $\lambda_n x_n \to 0$.  
    \begin{equation*}
        d(\lambda_n x_n, 0) = \sup_{i\in \N} |\xi|^{n/i} |a|^{1/i} \geq 1
    \end{equation*}  

    So multiplication by scalar is not continuous.
\end{eg}

\begin{defn}
    (Norm) Let $X$ be a nonempty set. A function $||\cdot||: X \to \R^+$ satisfying the following 
    is called a \textbf{norm}:
    \begin{itemize}
        \item (Positive definitiness) $\forall x \in X, ||x|| \geq 0$ and $||x||=0 \iff x=0$
        \item (Triangle-inequality) $\forall x,y \in X, ||x+y|| \leq ||x|| + ||y||$
        \item (Homogeneity) $\forall x \in X, \forall \lambda \in \mathbb{K}, ||\lambda x|| = |\lambda| ||x||$
    \end{itemize}
\end{defn}  

\begin{remark}
    Norm is a continuous function.
\end{remark}

\begin{defn}
(Normed Linear Spaces) A pair $(X, ||\cdot||)$ with $X$ being a linear space over $\mathbb{K}$ and 
                    $||\cdot||$ being a norm is called a \textbf{normed linear space}
\end{defn} 

Note that every normed linear space is a metric space, since every norm can induce a metric by $d(x,y)=||x-y||$. 
However, not every metric is a norm.  

\begin{eg}
    Let $X$ be the space of all sequences in $\R$ and $z>1$. A translation invariant metric $d$ is defined by  
    \begin{equation*}
        d(x,y) = \sum_{i=1}^{\infty} z^{-n} \dfrac{|x_i-y_i|}{1+|x_i-y_i|}
    \end{equation*}  
    But $d$ is not a norm, as it is not homogenous.
\end{eg}  

\begin{eg}
    Let $X=\R$ be the real numbers and $|\cdot|$ the Euclidean norm. Another example of a metric that is not a norm 
    is given by:
    \begin{equation*}
        d(x,y) = \min \{|x-y|, 1\}
    \end{equation*}  
    this is not a norm because it is not homogenous. (Note also that it is not translation invariant.)
\end{eg}  

\subsection{Common Spaces}  

\paragraph*{$l_p$ Spaces}    
For $p \in [1,\infty)$, the space $l_p$ is defined as the set of all sequences $(x_n)_{n\in \N}$ such that  

\begin{equation*}
    \sum_{n=1}^{\infty} |x_n|^p < \infty
\end{equation*}

the function  

\begin{equation*}
    ||x||_p = \left(\sum_{i=1}^n |x_i|^p\right)^{1/p}
\end{equation*}  

defines a norm on $l_p$.  

\paragraph*{$l_\infty$ Spaces}  
The space $l_\infty$ is defined as the set of all sequences $(x_n)_{n\in \N}$ such that

\begin{equation*}
    \sup_{n\in \N} |x_n| < \infty
\end{equation*}

the function

\begin{equation*}
    ||x||_\infty = \sup_{n\in \N} |x_n|
\end{equation*}

defines a norm on $l_\infty$.

\subsection{Inequalities}

\begin{prop}
(Young)
\end{prop}

\begin{cor}
(i)(Hölder)
(ii)(Minkowski)
\end{cor}

\begin{prop}
(Jensen)
\end{prop}

\begin{prop}
(Equivalent forms of Jensen)
\end{prop}  

\section{Completeness and Separability}  

\section{Hilbert Spaces}  

\section{Finite Dimensional Spaces}  

\section{Linear Operators}  

\section{Dual Spaces}  

\section{The Hahn Banach Theorems}  

\section{The Uniform Boundedness Theorem}
\subsection{Baire's Category Theorem}
\section{The Open Mapping Theorem}  

\section{The Closed Graph Theorem}  

\section{Compact Operators}
\end{document}
